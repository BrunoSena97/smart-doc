\chapter{Introduction} 
\label{chap:chap1}

Diagnostic error is one of the most persistent threats to patient safety, with
cognitive biases recognised as a major contributor to avoidable harm in clinical
practice~\parencite{graber_diagnostic_2005}. These biases—systematic deviations
from rational judgement—can derail even experienced clinicians, leading to
premature closure of reasoning, selective information use, or overconfidence in
a favoured hypothesis. As \textcite{croskerry_importance_2003} emphasises,
traditional training methods are often insufficient to help students recognise
and mitigate these cognitive pitfalls under the pressures of real-world practice.

At the same time, the transition from medical student to competent clinician
demands the development of robust clinical reasoning skills that combine
knowledge, pattern recognition, and reflective awareness~\parencite{audetat_diagnosis_2017}.
Conventional educational approaches emphasise outcomes, but often fail to make
the \emph{process} of reasoning visible, let alone directly train learners to
detect and counteract bias in their own thinking. Addressing this unmet need
requires pedagogical approaches that can elicit reasoning patterns, expose bias,
and scaffold reflection in a safe and scalable way.

This dissertation introduces \textbf{SmartDoc}, an AI-powered virtual patient
platform designed to meet this challenge. By harnessing large language models
(LLMs), SmartDoc provides naturalistic clinical interview simulations that not
only mirror the complexity of patient interaction, but also embed mechanisms to
detect cognitive biases in real time. Beyond replication of encounters, the
system delivers structured metacognitive feedback, prompting learners to reflect
on their reasoning. In this way, SmartDoc moves beyond teaching \emph{what} to
diagnose, toward teaching \emph{how} to think—directly addressing the root
causes of diagnostic error in a controlled, scalable environment
~\parencite{mamede_structure_2004,berge_cognitive_2013}.

\section{Motivation} 
\label{sec:se1}

The motivation for this research arises from two converging forces. First, an
educational gap: reasoning errors, rather than knowledge deficits, account for a
substantial proportion of diagnostic mistakes, yet medical training rarely
provides explicit support for bias awareness~\parencite{berge_cognitive_2013}.
Second, a technological opportunity: the rapid maturation of LLMs, together with
pedagogical frameworks from Intelligent Tutoring Systems (ITS), offers a new
toolkit for building simulations that combine realism with adaptive, reflective
support. SmartDoc is situated at the intersection of these forces: a system that
leverages AI to elicit naturalistic reasoning behaviour, surface bias-prone
moments, and scaffold reflective practice.

\section{Research Questions} 
\label{sec:rq}

The central research problem guiding this dissertation is: 

\begin{quote}
\textit{How can AI-powered virtual patients be designed to help medical students
recognise and mitigate cognitive biases in their diagnostic reasoning?}
\end{quote}

From this problem, three research questions (RQs) emerge:

\begin{itemize}
    \item \textbf{RQ1:} To what extent can an LLM-powered simulation realistically
    elicit and detect cognitive biases in diagnostic interviews?
    \item \textbf{RQ2:} How effective are metacognitive prompts—delivered in real
    time or post-hoc—in fostering reflection and reducing diagnostic bias?
    \item \textbf{RQ3:} What technical and pedagogical design principles enable
    the scalable deployment of bias-aware virtual patient simulations?
\end{itemize}

\section{Objectives} 
\label{sec:ae2}

In alignment with these questions, the dissertation pursues the following
objectives:

\begin{itemize}
\item \textbf{Simulation} --- To design and implement an AI-powered virtual
patient platform capable of conducting realistic, unscripted clinical interviews
based on authentic case scenarios engineered to elicit bias.
\item \textbf{Bias Detection} --- To embed mechanisms for identifying behavioural
markers of anchoring, confirmation bias, and premature closure in learner
interactions, using a combination of rule-based and LLM-supported approaches.
\item \textbf{Metacognitive Tutoring} --- To provide context-aware prompts that
stimulate reflection and encourage reconsideration of reasoning pathways when
bias-prone behaviour is detected.
\item \textbf{Evaluation Framework} --- To develop analytics that assess
diagnostic accuracy, information-gathering behaviours, and manifestations of
cognitive bias, enabling both formative and summative feedback.
\item \textbf{Effectiveness Study} --- To conduct a pilot evaluation of the
system with clinical interns, assessing usability, educational effectiveness,
and potential impact on bias awareness.
\end{itemize}

\section{Dissertation Structure} 
\label{sec:structure}

The remainder of this dissertation is organised as follows:

\begin{itemize}
    \item \textbf{Chapter 2} presents the theoretical background, outlining
    clinical reasoning, dual-process theory, and the taxonomy of cognitive
    biases, and introducing debiasing strategies and simulation principles that
    underpin SmartDoc’s design.
    \item \textbf{Chapter 3} reviews empirical evidence on AI-powered virtual
    patients through a scoping review, mapping feasibility, effectiveness,
    design considerations, and impact on reasoning, and identifying the gaps
    SmartDoc addresses.
    \item \textbf{Chapter 4} details the development of SmartDoc, showing how
    theoretical principles (Chapter 2) and empirical insights (Chapter 3)
    informed its architecture, case design, and bias-detection features.
    \item \textbf{Chapter 5} evaluates SmartDoc with clinical interns, presenting
    methods, results, and reflections on usability, realism, and educational
    impact.
    \item \textbf{Chapter 6} discusses the broader implications of the findings,
    acknowledges limitations, and outlines directions for future research and
    curriculum integration.
\end{itemize}

\noindent With the research problem, questions, and objectives established,
the next step is to ground the study in its theoretical foundations.
Chapter~\ref{chap:ch2} therefore examines the cognitive underpinnings of
clinical reasoning, the taxonomy of diagnostic biases, and existing debiasing
strategies, before introducing simulation—and particularly AI-powered virtual
patients—as a pedagogical approach that informs the design of SmartDoc.

